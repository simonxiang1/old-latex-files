\documentclass{article}
\usepackage{amsmath, amssymb, IEEEtrantools}
\usepackage{amsthm}

\title{Representation of Mathematical Expressions in \LaTeX}
\author{Simon Xiang}
\date{\today}
\theoremstyle{definition}
\newtheorem{definition}{Definition}

\begin{document}

\maketitle

\section{Limits}

\textbf{Example 1.1 }(Algebraic approach)\textbf{.} Consider the limit
\begin{equation*}
    \lim_{x \to 0} \frac{x}{3-\sqrt{x+9}}.
\end{equation*}
First, reduce the expression by completing the square, since directly plugging in the value for which
the limit converges into the equation will yield $0/0$, which is complete nonsense. 
\begin{equation*}
\begin{split}
    \frac{x}{3-\sqrt{x+9}}\quad&=\quad\frac{x(3+\sqrt{x+9})}{(3-\sqrt{x+9})(3+\sqrt{x+9})}\\
    &=\quad\frac{x(3+\sqrt{x+9})}{9-(\sqrt{x+9})^2}\\
    &=\quad\frac{x(3+\sqrt{x+9})}{-x}\\
    &=\quad-3-\sqrt{x+9}.\\
\end{split}
\end{equation*}
Therefore, we can safely say that the expressions $\frac{x}{3-\sqrt{x+9}}$ and $-3-\sqrt{x+9}$
are equivalent. Hence, the limit can be simply evaluated by plugging in the value that the limit
converges to into the second expression for $x$.
\begin{equation*}
\begin{split}
    \lim_{x \to 0} \frac{x}{3-\sqrt{x+9}}\quad&=\quad\lim_{x \to 0} -3-\sqrt{x+9}\\
    &=\quad-3-\sqrt{0+9}\\
    &=\quad-3-\sqrt{9}\\
    &=\quad-6\\
\end{split}
\end{equation*}
Thus
\begin{equation*}
    \lim_{x \to 0} \frac{x}{3-\sqrt{x+9}}\quad=\quad-6.
\end{equation*}

\noindent\textbf{Example 1.2 }(L'hopital's Rule)\textbf{.}
Assuming knowledge of L'Hopital's Rule, one basic limit that first-year Calculus students
are often told to memorize is
\begin{equation*}
    \lim_{x \to 0} \frac{\sin (x)}{x} = 1.
\end{equation*}
However, with a working knowledge of derivatives and L'Hopital's rule, we can prove that 
the value of this limit is indeed $1$. First, let us state L'Hopital's rule.

\begin{definition}{\textit{(L'Hopital's rule).}}
    Let $f(x)$ and $g(x)$ be differentiable functions where $g(x)\neq0$. If
\begin{equation*}
    \lim_{x \to c} \frac{f(x)}{g(x)} = L \quad\ni\quad c, L \in \mathbb{R}\cup\{-\infty,+\infty\},
\end{equation*}
and either of
\begin{equation*}
    \lim_{x \to c} f(x)=\lim_{x \to c} g(x)=0,\quad\lim_{x \to c} |f(x)|=\lim_{x \to c} |g(x)|=\infty
\end{equation*}
are true, then
\begin{equation*}
    \lim_{x \to c} \frac{f(x)}{g(x)}\quad=\quad\lim_{x \to c} \frac{f'(x)}{g'(x)}\quad=\quad L.
\end{equation*}
\end{definition}

If we let $f(x)=\sin(x)$ and $g(x)=x$, then evaluating $\lim_{x \to 0} \sin(x)$ and $\lim_{x \to 0} x$
both yield $0$ as the limit, therefore the function is in the perfect form to directly apply L'Hopital's rule.
Let us do so.
\begin{equation*}
\begin{split}
    \lim_{x \to 0} \frac{\sin(x)}{x}\quad&=\quad\lim_{x \to 0} \frac{\frac{d}{dx}\sin(x)}{\frac{d}{dx}x}\\
    &=\quad\lim_{x \to 0} \frac{\cos(x)}{1}\\
    &=\quad\frac{\cos(0)}{1}\\
    &=\quad1\\
\end{split}
\end{equation*}
Therefore we have verified using L'Hopital's rule that the limit
\begin{equation*}
    \lim_{x \to 0} \frac{\sin (x)}{x} = 1.
\end{equation*}
is indeed true.

\section{Derivatives}
\noindent\textbf{Example 2.1 }(Chain, product, and quotient rule)\textbf{.}
In a moment of genius, I realized I would only have to type one example instead of three if I used 
a derivative problem that involved the product, quotient, \textit{and} the chain rule!

\begin{definition}{\textit{(Product rule).}}
Let $f(x)=g(x)\cdot h(x)$, where $g(x)$ and $h(x)$ are both differentiable functions of $x$. Then
\begin{equation*}
    f'(x)\quad=\quad (g'(x)\cdot h(x)) + (h'(x)\cdot g(x)).
\end{equation*}
\end{definition}

\begin{definition}{\textit{(Quotient rule).}}
Let $f(x)=\frac{g(x)}{h(x)}$, where $g(x)$ and $h(x)$ are both differentiable functions of $x$, %
and $h(x)\neq0$. Then
\begin{equation*}
    f'(x)\quad=\quad \frac{(g'(x)\cdot h(x)) - (h'(x)\cdot g(x))}{[h(x)]^2}.
\end{equation*}
\end{definition}

\begin{definition}{\textit{(Chain rule, standard).}}
Let $f(x)=g(h(x))$, where $g(x)$ and $h(x)$ are both differentiable functions of $x$. Then
\begin{equation*}
    f'(x)\quad=\quad g'(h(x))\cdot h'(x).
\end{equation*}
\end{definition}

\begin{definition}{\textit{(Chain rule, composition of functions).}}
Let $F = f\circ g$, then
\begin{equation*}
    F'\quad=\quad (f'\circ g)\cdot g'.
\end{equation*}
\end{definition}

\begin{definition}{\textit{(Chain rule, Leibniz notation).}}
If we have a function such that the variable $y$ depends on the variable $u$, and $u$ depends on the %
variable $x$, we can safely say that $y$ depends on $x$, and thus 
\begin{equation*}
    \frac{dy}{dx} = \frac{dy}{du}\cdot\frac{du}{dx}.\footnote{I omitted doing two problems because of %
    laziness yet defined the chain rule three times. Ironic.}
\end{equation*}
\end{definition}
Now that we have all the tools in our toolbox, let us begin to evaluate a particularly nasty derivative. %
Let
\begin{equation*}
    f(x)\quad=\quad\frac{\sin(x^2)\sqrt{e^x+e^{-x}}}{4^x}.
\end{equation*}
To solve for $f'(x)$, let $g(x)=\sin(x^2)\sqrt{e^x+e^{-x}}$ and $h(x)=4^x$. Apply the quotient rule to get
\begin{equation*}
    f'(x)\quad=\quad\frac{\Big(\frac{d}{dx}\big(\sin(x^2)\sqrt{e^x+e^{-x}}\big)\cdot %
    4^x\Big)-\Big(\frac{d}{dx}(4^x)\cdot sin(x^2)\sqrt{e^x+e^{-x}}\Big)}{(4^x)^2} 
\end{equation*}
To finish solving out the problem, we will need to compute $\frac{d}{dx}\big(\sin(x^2)\sqrt{e^x+e^{-x}}\big)$
which requires both the product and the chain rule, and $\frac{d}{dx}(4^x)$, which also requires the chain and 
product rule. Let us compute the easier derivative first, recalling that $e^{\ln(x)}=x$.
\begin{equation*}
    \overline f(x) = 4^x = e^{ln(4^x)}.
\end{equation*}
Let $u(x) = ln(4^x) = xln(4)$, and therefore $\overline f(u) = e^u$. Set $\overline f = y$, and apply the %
Leibniz definition of the chain rule. Recall that $\frac{d}{dx}e^x = e^x$ to finish solving for %
$\overline f'(x)$.
\begin{equation*}
    \frac{dy}{du} = e^u,\quad \frac{du}{dx} = ln(4),\quad\frac{dy}{dx} = \frac{dy}{du}\cdot %
    \frac{du}{dx} = e^u \cdot ln(4) = e^{ln(4^x)}ln(4) = 4^x ln(4)
\end{equation*}
Let us define another function $\hat{f}(x) = \sin(x^2)\sqrt{e^x+e^{-x}}$. If we set $\hat{g}(x) = \sin(x^2)$ %
and $\hat{h}(x) = \sqrt{e^x+e^{-x}}$, we can continue on to apply the product rule. To solve for $\hat{g}'(x)$
and $\hat{h}'(x)$, the chain rule will need to be applied.
\begin{equation*}
\begin{split}
    \hat{f}'(x) &= \hat{g}'(x)\cdot \hat{h}(x) + \hat{h}'(x)\cdot \hat{g}(x)\\
    \hat{g}'(x) &= \cos(x^2)\cdot \frac{d}{dx}(x^2) = 2x\cos(x^2)\\
    \hat{h}'(x) &= \frac{1}{2\sqrt{e^x+e^{-x}}}\cdot \frac{d}{dx}(e^x+e^{-x}) = %
    \frac{e^x-e^{-x}}{2\sqrt{e^x+e^{-x}}}\\
    \hat{f}'(x) &= \big(2x\cos(x^2)\cdot \sqrt{e^x+e^{-x}}\big) + %
    \big(\frac{e^x-e^{-x}}{2\sqrt{e^x+e^{-x}}}\cdot \sin(x^2)\big)
\end{split}
\end{equation*}
Substitute this new information into the original equation to get
\begin{equation*}
\begin{split}
    f'(x) &= \frac{\big(\hat{f}'(x)\cdot \overline f(x)\big) - %
    \big(\overline f'(x)\cdot \hat{f}(x)\big)}{(\overline f(x))^2}\\
    &=\frac{\bigg[4^x\bigg(\Big(2x\cos(x^2)\sqrt{e^x+e^{-x}}\Big) + %
    \Big(\frac{e^x-e^{-x}}{2\sqrt{e^x+e^-x}}\cdot \sin(x^2)\Big)\bigg)\bigg] - %
    \bigg[ln(4)\cdot 4^x \Big(\sin(x^2)\cdot \sqrt{e^x+e^{-x}}\Big)\bigg]}{4^{2x}}\\
    &=\frac{\Big((2x\cos(x^2)\sqrt{e^x+e^{-x}}\Big) + %
    \Big(\frac{e^x-e^{-x}}{2\sqrt{e^x+e^-x}}\cdot \sin(x^2)\Big) - %
    \Big(ln(4)\cdot 4^x \big(\sin(x^2)\cdot \sqrt{e^x+e^{-x}}\big)\Big)}{4^x}\\
    &=\frac{2x\cos(x^2)\sqrt{e^x+e^{-x}}}{4^x} + %
    \frac{(e^x-e^{-x})\sin(x^2)}{2\cdot 4^x\sqrt{e^x+e^{-x}}} - %
    \frac{ln(4)\sin(x^2)\sqrt{e^x+e^{-x}}}{4^x}.
\end{split}
\end{equation*}

\noindent\textbf{Example 2.2 }(Partial derivatives and the chain rule)\textbf{.}
The multivariate chain takes on a different, yet familiar form when juxtaposed to its 
single-variable counterpart.
\begin{definition}{\textit{(Multivariate chain rule).}}
Let $z = f(x,y)$, where $x$ and $y$ and both functions of $t$. If $z$ is %
differentiable at $x(t)$, $y(t)$, then
\begin{equation*}
    \frac{dz}{dt} = \frac{\partial z}{\partial x}\cdot \frac{dx}{dt} + %
    \frac{\partial z}{\partial y}\cdot \frac{dy}{dt}.
\end{equation*}
\end{definition}
\noindent Jumping right into the problem, find $\frac{dy}{dx}$ of $x^2\tan(5y) + x^3y^3 %
= 4x - e^{(x^2+y^2)}$. Hold up. This looks awfully familiar to a concept learned in
a first-semester Calculus course and nothing like a question that involves partial
derivatives! However, we can rewrite this as a multivariate function in the form
$z = f(x,y)$, where $y = \hat{f}(x)$. Therefore, if we wanted to find $\frac{dz}%
{dx}$, we could apply the multivariate chain rule and obtain the following:
\begin{equation*}
\begin{split}
    \frac{dz}{dx} &= \frac{\partial f}{\partial x}\cdot \frac{dx}{dx} + \frac{\partial f}%
    {\partial y}\cdot \frac{dy}{dx} = \frac{\partial f}{\partial x} + \frac{\partial f}%
    {\partial y}\cdot \frac{dy}{dx}\\
    \frac{dy}{dx} &= -\frac{\big(\frac{\partial f}{\partial x}\big)}%
    {\big(\frac{\partial f}{\partial y}\big)} = -\frac{f_x}{f_y}
\end{split}
\end{equation*}
This equation is known as the \textit{Implicit Function Theorem}, and we can apply it here
to take the implicit derivative of the given function. Rearrage the equation as such to
obtain a zero on one side: $x^2\tan(5y) + x^3y^3 - 4x + e^{x^2+y^2} = 0$. Now we just have
to solve for $f_x$ and $f_y$ and plug it into our newly obtained formula to finish the question.
\begin{equation*}
\begin{split}
    f_x &= 2x\tan(5y) + 3x^2y^3 - 4 + 2xe^{(x^2+y^2)}\\
    f_y &= 5x^2\sec^2(5y) + 3y^2x^3 + 2ye^{(x^2+y^2)}\\
    \frac{dy}{dx} &= -\frac{f_x}{f_y} = -\frac{2x\tan(5y) + 3x^2y^3 - 4 + 2xe^{(x^2+y^2)}}%
    {5x^2\sec^2(5y) + 3y^2x^3 + 2ye^{(x^2+y^2)}}
\end{split}
\end{equation*}

\section{Integrals}
\noindent\textbf{Example 3.1 }(Difficult integration)\textbf{.}
I was going to make separate sections for matrix operations and integration, but in yet
another stroke of genius I realized we can just solve integration problems using basic
linear algebra! Here we attempt an infamous integral.
\begin{equation*}
    \int\!\sqrt{\tan(x)}\,dx
\end{equation*}
The substitution to be used is not immediately clear, and the later part of this integral
involves a particularly nasty case of partial fractions, which makes this such a 
challenging integral. Jumping right into the question, use the substitution $u = \sqrt{\tan(x)}$
to reduce the integral to something that partial fractions can be applied on.
\begin{equation*}
\begin{split}
    u = \sqrt{\tan(x)} \qquad u^2 = \tan(x)\\
    2u\,du = \sec^2(x)\,dx \qquad dx = \frac{2u\,du}{\sec^2(x)} \\
\end{split}
\end{equation*}
\begin{equation*}
\begin{gathered}
    \int\!\sqrt{\tan(x)}\,dx = \int (u)\cdot \big(\frac{2u\,du}{\sec^2(x)}\big) 
    = \int\!\frac{2u^2\,du}{\sec^2(x)-1+1} = \int\!\frac{2u^2\,du}{\tan^2(x)+1}\\
    = \int\!\frac{2u^2\,du}{u^4+1}
\end{gathered}
\end{equation*}

Now the problem remains of how to factor a polynomial in the form $x^4+1$, which involves a 
sneaky complete the square operation to put the polynomial into a form that allows us to apply
the difference of squares formula to factor it. Truly, this question is quite difficult and
probably should not be assigned as homework for an introductory calculus course (looking at my
Cal II professor).
\begin{equation*}
\begin{gathered}
    u^4+1\quad=\quad u^4+2u^2+1-2u^2\quad=\quad (u^2+1)-(\sqrt{2}u)^2\\
    =(u^2+\sqrt{2}u+1)(u^2-\sqrt{2}u+1)
\end{gathered}
\end{equation*}
Now we have to decompose $\frac{2u^2}{(u^2+\sqrt{2}u+1)(u^2-\sqrt{2}u+1)}$ by partial fractions
which will leave a rather troublesome system of linear equations, thankfully we can solve
it by simply inverting the resultant matrix to solve for the vector $x$, which can be represented 
as such: $A\vec{x}=\vec{b} \qquad \vec{x}=A^{-1}\vec{b}$
\begin{equation*}
\begin{gathered}
    \frac{2u^2}{(u^2+\sqrt{2}u+1)(u^2-\sqrt{2}u+1)}  =  \frac{Au+B}{(u^2+\sqrt{2}u+1)}
    + \frac{Cu+D}{(u^2-\sqrt{2}u+1)}\\
    = (Au+B)(u^2-\sqrt{2}u+1)+(Cu+D)(u^2+\sqrt{2}u+1)\\
    = (Au^3-\sqrt{2}Au^2+Au) + (Bu^2-\sqrt{2}Bu+B) + (Cu^3+\sqrt{2}Cu^2+Cu) + (Du^2+\sqrt{2}Du+D)\\
    = u^3(A+C)+u^2(B-\sqrt{2}A+\sqrt{2}C+D)+u(A-\sqrt{2}B+\sqrt{2}D+C)+(B+D)\\
    \text{Let}\quad u=0, \quad \text{then} \quad \boxed{(B+D)=0}\\
    \frac{2u^2}{u} = \frac{u^3(A+C)+u^2(B-\sqrt{2}A+\sqrt{2}C+D)+u(A-\sqrt{2}B+\sqrt{2}D+C)}{u}\\
    2u = u^2(A+C)+u(B-\sqrt{2}A+\sqrt{2}C+D)+(A-\sqrt{2}B+\sqrt{2}D+C)\\
    \text{Let}\quad u=0, \quad \text{then} \quad \boxed{(A-\sqrt{2}B+\sqrt{2}D+C)=0}\\
    \frac{2u}{u} = \frac{u^2(A+C)+u(B-\sqrt{2}A+\sqrt{2}C+D)}{u}\\
    2 = u(A+C)+(B-\sqrt{2}A+\sqrt{2}C+D)\\
    \text{Let}\quad u=0, \quad \text{then} \quad \boxed{(B-\sqrt{2}A+\sqrt{2}C+D)=2}\\
    2 = u(A+C)+2\\
    (A+C) = \frac{0}{u} = 0\\
    \boxed{(A+C)=0}
\end{gathered}
\end{equation*}
To complete the partial fraction decomposition, all that's left is to solve the resultant system
of linear equations. Arrange the equations as such to make it easier to set up the matrix equation:
\begin{equation*}
\begin{array}{lclclclcl}
    A & + & 0 & + & C & + & 0 & = & 0\\
    -\sqrt{2}A & + & B & + & \sqrt{2}C & + & D & = & 2\\
    A & + & -\sqrt{2}B & + & C & + & \sqrt{2}D & = & 0\\
    0 & + & B & + & 0 & + & D & = & 0  
\end{array}
\end{equation*}
Now rewrite this system as a matrix and invert the matrix to finish solving for $A, B, C$ and $D$.
\[
\begin{gathered}
A = \begin{bmatrix}
    1 & 0 & 1 & 0\\
    -\sqrt{2} & 1 & \sqrt{2} & 1\\
    1 & -\sqrt{2} & 1 & \sqrt{2}\\
    0 & 1 & 0 & 1
\end{bmatrix}, \qquad
\vec{x} = \begin{bmatrix}
    A\\
    B\\
    C\\
    D
\end{bmatrix}, \qquad
\vec{b} = \begin{bmatrix}
    0\\
    2\\
    0\\
    0
\end{bmatrix}\\
A\vec{x} = \vec{b}, \qquad \vec{x} = A^{-1}\vec{b}
\end{gathered}
\]
Now we could have solved this system with row reduction, but I am an avid hater of row reduction,
and so we invert the matrix to solve the system instead. By extension, we're not calculating this
inverse with the $[A\,|\,I]\quad \overrightarrow{row\,reduction} \quad[I\,|\,A^{-1}]$ method, but
using the method involving determinants and the adjugate matrix instead, where
\begin{equation*}
    A^{-1} = \frac{1}{det(A)}adj(A).
\end{equation*}
In order to calculate the determinant of the matrix, perform a cofactor expansion along the first 
row, then along the third row for the resultant two matrices. 
\[
\begin{gathered}
\begin{vmatrix}
    1 & 0 & 1 & 0\\
    -\sqrt{2} & 1 & \sqrt{2} & 1\\
    1 & -\sqrt{2} & 1 & \sqrt{2}\\
    0 & 1 & 0 & 1
\end{vmatrix}
= 1\cdot
\begin{vmatrix}
    1 & \sqrt{2} & 1\\
    -\sqrt{2} & 1 & \sqrt{2}\\
    1 & 0 & 1
\end{vmatrix} + 1\cdot
\begin{vmatrix}
    -\sqrt{2} & 1 & 1\\
    1 & -\sqrt{2} &\sqrt{2}\\
    0 & 1 & 1
\end{vmatrix}\\
= \left(1 \cdot \begin{vmatrix}
    \sqrt{2} & 1\\
    1 & \sqrt{2}
\end{vmatrix} + 1\cdot \begin{vmatrix}
    1 & \sqrt{2}\\
    -\sqrt{2} & 1
\end{vmatrix}\right) + \left(-1\cdot\begin{vmatrix}
    -\sqrt{2} & 1\\
    1 & \sqrt{2}
\end{vmatrix} + 1\cdot \begin{vmatrix}
    -\sqrt{2} & 1\\
    1 & -\sqrt{2}
\end{vmatrix}\right)\\
=\left(1+3\right) + \left(3+1\right)\\
=8
\end{gathered}
\]
To compute the adjugate matrix, we would need to calculate fourteen more 3x3 determinants for each
coordinate of the matrix to complete the matrix of minors. However, such computations are trivial
and left to the reader as an exercise. Thus, we have the matrix of minors as such:
\begin{equation*}
\begin{bmatrix}
    4 & -2\sqrt{2} & 4 & 2\sqrt{2}\\
    2\sqrt{2} & 0 & -2\sqrt{2} & 0\\
    0 & 2\sqrt{2} & 0 & -2\sqrt{2}\\
    -2\sqrt{2} & 4 & 2\sqrt{2} & 4
\end{bmatrix}
\end{equation*}
Multiply the matrix of minors by the ``checkerboard'' pattern to find the matrix of cofactors, then
take the transpose of the resulting matrix to get the adjugate matrix of $A$.
\[
\begin{gathered}
\begin{bmatrix}
    4 & -2\sqrt{2} & 4 & 2\sqrt{2}\\
    2\sqrt{2} & 0 & -2\sqrt{2} & 0\\
    0 & 2\sqrt{2} & 0 & -2\sqrt{2}\\
    -2\sqrt{2} & 4 & 2\sqrt{2} & 4
\end{bmatrix} \quad \Longrightarrow \quad \begin{bmatrix}
    4 & 2\sqrt{2} & 4 & -2\sqrt{2}\\
    -2\sqrt{2} & 0 & 2\sqrt{2} & 0\\
    0 & -2\sqrt{2} & 0 & 2\sqrt{2}\\
    2\sqrt{2} & 4 & -2\sqrt{2} & 4
\end{bmatrix}\\
\begin{bmatrix}
    4 & 2\sqrt{2} & 4 & -2\sqrt{2}\\
    -2\sqrt{2} & 0 & 2\sqrt{2} & 0\\
    0 & -2\sqrt{2} & 0 & 2\sqrt{2}\\
    2\sqrt{2} & 4 & -2\sqrt{2} & 4
\end{bmatrix}^T \quad = \quad \begin{bmatrix}
    4 & -2\sqrt{2} & 0 & 2\sqrt{2}\\
    2\sqrt{2} & 0 & -2\sqrt{2} & 4\\
    4 & 2\sqrt{2} & 0 & -2\sqrt{2}\\
    -2\sqrt{2} & 0 & 2\sqrt{2} & 4
\end{bmatrix}
\end{gathered}
\] 
Now that we have both the adjugate matrix and the determinant, we can divide the adjugate matrix
by the determinant of $A$ to get $A^{-1}$.
\[
\begin{gathered}
    A^{-1} = \frac{1}{det(A)}adj(A) = \frac{\begin{bmatrix}
        4 & -2\sqrt{2} & 0 & 2\sqrt{2}\\
        2\sqrt{2} & 0 & -2\sqrt{2} & 4\\
        4 & 2\sqrt{2} & 0 & -2\sqrt{2}\\
        -2\sqrt{2} & 0 & 2\sqrt{2} & 4
    \end{bmatrix}}{8}\\
    =\begin{bmatrix}
        \frac{1}{2} & -\frac{1}{2\sqrt{2}} & 0 & \frac{1}{2\sqrt{2}}\\
        \frac{1}{2\sqrt{2}} & 0 & -\frac{1}{2\sqrt{2}} & \frac{1}{2}\\
        \frac{1}{2} & \frac{1}{2\sqrt{2}} & 0 & -\frac{1}{2\sqrt{2}}\\
        -\frac{1}{2\sqrt{2}} & 0 & \frac{1}{2\sqrt{2}} & \frac{1}{2}
    \end{bmatrix}
\end{gathered}
\]
From here, all that remains is to multiply the inverted matrix by the vector $\vec{b}$ to obtain
the solution vector $\vec{x}$.
\begin{equation*}
\begin{gathered}
\vec{x} = A^{-1}\vec{b} = \begin{bmatrix}
    \frac{1}{2} & -\frac{1}{2\sqrt{2}} & 0 & \frac{1}{2\sqrt{2}}\\
    \frac{1}{2\sqrt{2}} & 0 & -\frac{1}{2\sqrt{2}} & \frac{1}{2}\\
    \frac{1}{2} & \frac{1}{2\sqrt{2}} & 0 & -\frac{1}{2\sqrt{2}}\\
    -\frac{1}{2\sqrt{2}} & 0 & \frac{1}{2\sqrt{2}} & \frac{1}{2}
\end{bmatrix}\cdot \begin{bmatrix}
    0\\
    2\\
    0\\
    0
\end{bmatrix} = \begin{bmatrix}
    -\frac{1}{2\sqrt{2}}\\
    0\\
    \frac{1}{2\sqrt{2}}\\
    0
\end{bmatrix} \cdot 2 = \begin{bmatrix}
    -\frac{1}{\sqrt{2}}\\
    0\\
    \frac{1}{\sqrt{2}}\\
    0
\end{bmatrix}\\
\begin{bmatrix}
    A\\
    B\\
    C\\
    D
\end{bmatrix} = \begin{bmatrix}
    -\frac{1}{\sqrt{2}}\\
    0\\
    \frac{1}{\sqrt{2}}\\
    0
\end{bmatrix} \qquad A = -\frac{1}{\sqrt{2}} \qquad B = 0 \qquad C = \frac{1}{\sqrt{2}} %
\qquad D = 0\\
\frac{Au+B}{(u^2+\sqrt{2}u+1)} + \frac{Cu+D}{(u^2-\sqrt{2}u+1)} = %
\frac{-\frac{1}{\sqrt{2}}u+0}{(u^2+\sqrt{2}u+1)}
+ \frac{\frac{1}{\sqrt{2}}u+)}{(u^2-\sqrt{2}u+1)}\\
=\boxed{\frac{u}{\sqrt{2}(u^2-\sqrt{2}u+1)} - \frac{u}{\sqrt{2}(u^2+\sqrt{2}u+1)}}
\end{gathered}
\end{equation*}
To finish the problem, simply integrate the newly obtained expression and plug back in 
$\sqrt{\tan(x)}$ for $u$. Integrating this expression requires splitting the integral
so that the left half can be integrated with a simple u-sub and the right half with a 
complete the square, which looks like this: $u = \frac{1}{2}(2u-\sqrt{2})+\frac{\sqrt{2}}{2}$
for the left portion and $u = \frac{1}{2}(2u+\sqrt{2})-\frac{\sqrt{2}}{2}$ for the right.
\begin{equation*}
\begin{gathered}
    \int\! \left(\frac{u}{\sqrt{2}(u^2-\sqrt{2}u+1)} - \frac{u}{\sqrt{2}(u^2+\sqrt{2}u+1)}\right)\,du\\
\end{gathered}
\end{equation*}
\begin{equation*}
\begin{gathered}
    =\frac{1}{\sqrt{2}}\int\! \left(\frac{\frac{1}{2}(2u-\sqrt{2})+\frac{\sqrt{2}}{2}}{\sqrt{2}(u^2-\sqrt{2}u+1)}\right)\,du - %
    \frac{1}{\sqrt{2}}\int\! \left(\frac{\frac{1}{2}(2u+\sqrt{2})-\frac{\sqrt{2}}{2}}{\sqrt{2}(u^2+\sqrt{2}u+1)}\right)\,du\\
    =\frac{1}{2\sqrt{2}}\int\! \left(\frac{2u-\sqrt{2}}{(u^2-\sqrt{2}u+1)}\right)\,du + %
    \frac{1}{2}\int\! \left(\frac{1}{(u^2-\sqrt{2}u+1)}\right)\,du  \\
    -\frac{1}{2\sqrt{2}}\int\! \left(\frac{2u+\sqrt{2}}{(u^2+\sqrt{2}u+1)}\right)\,du + %
    \frac{1}{2}\int\! \left(\frac{1}{(u^2+\sqrt{2}u+1)}\right)\,du\\
    \textrm{Let $v=u^2-\sqrt{2}u+1,\quad dv = (2u-\sqrt{2}) \,du$ \qquad and \qquad $w=u^2+\sqrt{2}u+1, %
    \quad dw=(2u+\sqrt{2}) \,du$}\\
    =\frac{1}{2\sqrt{2}}\int\! \left(\frac{dv}{v}\right) + %
    \frac{1}{2}\int\! \left(\frac{1}{(u-\frac{\sqrt{2}}{2})^2+\frac{1}{2}}\right)\,du
    -\frac{1}{2\sqrt{2}}\int\! \left(\frac{dw}{w}\right) + %
    \frac{1}{2}\int\! \left(\frac{1}{(u+\frac{\sqrt{2}}{2})^2+\frac{1}{2}}\right)\,du\\
    \textrm{Let $\hat{v}=u-\frac{\sqrt{2}}{2},\quad d\hat{v} = du$ \qquad and \qquad %
    $\hat{w}=u+\frac{\sqrt{2}}{2},\quad d\hat{w} = du$}\\
    =\frac{1}{2\sqrt{2}}\ln(v) + \frac{1}{2}\int\! \left(\frac{d\hat{v}}{\hat{v}^2+\frac{1}{2}}\right)
    -\frac{1}{2\sqrt{2}}\ln(w) + \frac{1}{2}\int\! \left(\frac{d\hat{w}}{\hat{w}^2+\frac{1}{2}}\right) + C\\
    =\frac{1}{2\sqrt{2}}\ln(v) - \frac{1}{2\sqrt{2}}\ln(w) + %
    \frac{1}{2}\int\! 2\left(\frac{d\hat{v}}{(\sqrt{2}\hat{v})^2+1}\right) +
    \frac{1}{2}\int\! 2\left(\frac{d\hat{v}}{(\sqrt{2}\hat{w})^2+1}\right) + C\\
    =\frac{1}{2\sqrt{2}}\ln(v) - \frac{1}{2\sqrt{2}}\ln(w) + 
    \frac{\sqrt{2}}{2}\arctan(\sqrt{2}\hat{v}) + \frac{\sqrt{2}}{2}\arctan(\sqrt{2}\hat{w}) + C
\end{gathered}
\end{equation*}
The integral is almost finished! All that remains is to plug back in the dummy variables used to 
obtain our final answer.
\begin{equation*}
\begin{gathered}
    \frac{1}{2\sqrt{2}}\ln(v) - \frac{1}{2\sqrt{2}}\ln(w) + 
    \frac{\sqrt{2}}{2}\arctan(\sqrt{2}\hat{v}) + \frac{\sqrt{2}}{2}\arctan(\sqrt{2}\hat{w}) + C\\
    =\frac{1}{2\sqrt{2}}\ln(u^2-\sqrt{2}u+1) - \frac{1}{2\sqrt{2}}\ln(u^2+\sqrt{2}u+1) + \\
    \frac{\sqrt{2}}{2}\arctan\left(\sqrt{2}\left(u-\frac{\sqrt{2}}{2}\right)\right) + %
    \frac{\sqrt{2}}{2}\arctan\left(\sqrt{2}\left(u+\frac{\sqrt{2}}{2}\right)\right) + C\\
    =\frac{1}{2\sqrt{2}}\ln\left(\tan(x)-\sqrt{2\tan(x)}+1\right) - %
    \frac{1}{2\sqrt{2}}\ln\left(\tan(x)+\sqrt{2\tan(x)}+1\right) + \\
    \frac{\sqrt{2}}{2}\arctan\left(\sqrt{2\tan(x)}-1\right) + %
    \frac{\sqrt{2}}{2}\arctan\left(\sqrt{2\tan(x)}+1\right) + C\\
\end{gathered}
\end{equation*}
\tiny
\begin{equation*}
\begin{gathered}
    =\boxed{\frac{\ln\left(\tan(x)-\sqrt{2\tan(x)}+1\right) - \ln\left(\tan(x)+\sqrt{2\tan(x)}+1\right) %
    + 2\arctan\left(\sqrt{2\tan(x)}-1\right) + 2\arctan\left(\sqrt{2\tan(x)}+1\right)}{2\sqrt{2}} + C}
\end{gathered}
\end{equation*}
\normalsize
\end{document}
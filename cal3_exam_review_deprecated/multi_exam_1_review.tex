         %_
 %___ ___| |__
%/ __/ __| '_ \
%\__ \__ \ | | |
%|___/___/_| |_|

\documentclass{article}
\usepackage{amsmath, amssymb, IEEEtrantools}
\usepackage{amsthm}

\title{Calculus III Review}
\author{Simon Xiang}
\date{\today}
\theoremstyle{definition}
\newtheorem{definition}{Definition}
\newtheorem{thm}{Theorem}

\begin{document}

\maketitle

\section{Vectors}
Vectors are elements of something called a \textit{vector space}, which in essence are subsets of $\mathbb{R}^n$.
A typical vector can be notated several ways:
\begin{equation}
    \langle a,b,c \rangle \quad = \quad a\boldsymbol{\hat{i}} + b\boldsymbol{\hat{j}} + c\boldsymbol{\hat{k}}
\end{equation}
are the two most typical examples of how they're notated. $\boldsymbol{\hat{i}},\boldsymbol{\hat{j}},$ and $\boldsymbol{\hat{k}}$
are called the \textit{unit vectors} of $\mathbb{R}^3$. This is because every vector is just the unit vectors multiplied
by some scalar quantity, hence the notation. In fact, the entirety of $\mathbb{R}^3$ can be expressed through adding and
multiplying the unit vectors, something known as a \textit{linear combination} of $\boldsymbol{\hat{i}},\boldsymbol{\hat{j}},$ and $\boldsymbol{\hat{k}}$.

The vector $\vec{PQ}$ representing the distance betweent two points $P, Q \in \mathbb{R}^3$ where $P = (p_1,p_2,p_3)$ and $Q = 
(q_1,q_2,q_3)$ is equal to 
\begin{equation}
    \langle q_1-p_1, \, q_2-p_2, \, q_3-p_3 \rangle , 
\end{equation}
 where $P$ is the base of the vector and $Q$ is the tip of the vector. A vector in the form $\langle a,b,c \rangle$ is the same
 as the vector between $(0,0,0)$ and $(a,b,c)$. 

 Where the vector is placed in $\mathbb{R}^3$ is irrelevant. Let $P=(1,2,3)$ and $Q=(4,6,5), (a,b,c)=(3,4,2)$. Then 
 $\vec{PQ}$ is equivalent to $\langle a,b,c \rangle$, even though they lie in different "parts" of the plane.

 The length or magnitude of a vector is defined by 
\begin{equation}
  \lvert \vec{PQ} \rvert = \sqrt{(q_1-p_1)^2 + (q_2-p_2)^2 + (q_3-p_3)^2}.    
\end{equation}
This formula can be thought of as the distance between two far corners of a cube, where one of the edges of a right 
triangle is $\sqrt{a^2+b^2}$ and the other is $c$. This intuition is helpful when we eventually 
talk about the arc length of curves. 

A vector divided by its length will give a vector pointing in the same direction as the original vector, but having 
a length of one. This is called the \textit{directional unit vector} of a vector $\vec{v}$, and is given by 
\begin{equation}
    \frac{\vec{v}}{\lvert v \rvert}.
\end{equation}


\end{document}

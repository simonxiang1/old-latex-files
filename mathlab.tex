\documentclass{article}
\usepackage[utf8]{inputenc}

\begin{document}
\noindent
\textbf{Question 1: Why are you interested in this position and why do you feel that you would be a good choice?}

I am interested in this position because I enjoy doing mathematics, and making money while tutoring students is very appealing to me. I already spend a large portion of my free time tutoring my peers, so the work won’t be too stressful since I’m experienced and enjoy doing it. Furthermore, working here would strengthen and maintain my knowledge in lower level math courses like Calculus and Linear Algebra, better preparing me for grad school. I believe I would make a good choice because I am experienced in tutoring. I also have taken several higher-level math courses like Real Analysis and Abstract Algebra and feel confident in my ability to teach such topics. 

\vspace{3mm}
\noindent
\textbf{Question 2: What is the most important quality of an effective tutor and why?}

I believe the most important quality of an effective tutor is the ability to break complex topics into simple language. Math terminology can get overwhelming quickly, and when students get overwhelmed their brains tend to shut down a little and they begin to tune out what they’re listening to. Being able to explain harder concepts with simple language demonstrates a strong grasp on the topic, as well as effective tutoring skills. Those being tutored will understand more and feel more comfortable with the topic if difficult questions and concepts are phrased simply by the tutor. 

\vspace{3mm}
\noindent
\textbf{Question 3: You are working with a MATH 1100 student and notice that three other MATH 1100 students are waiting to be helped. What would you do in this situation? }

A lot of the times when I help someone with math, I’ll ask them to stop at a certain point and figure out a concept by themselves, with guided hints. If I notice three students waiting for help while I’m helping one student, I would probably find a good stopping point and ask them to work the rest of the problem/figure out the rest of the proof themselves with some hints, and move over to help the three that have been waiting. After I finish helping the three people, I would return to the one student and ask how their progress has been going, and direct them in the right direction if needed. 

\end{document}



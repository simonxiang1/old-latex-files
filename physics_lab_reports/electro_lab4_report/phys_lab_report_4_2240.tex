\documentclass{article}
\usepackage{amsmath, amssymb, IEEEtrantools}
\usepackage{amsthm}
\usepackage{graphicx}
\usepackage{array}
\usepackage{circuitikz}
\usepackage{float}
\usepackage{caption}
\newcommand*{\equal}{=}

\title{Representation of Mathematical Expressions in \LaTeX}
\author{Simon Xiang}
\date{\today}
\theoremstyle{definition}
\newtheorem{definition}{Definition}

\begin{document}
\begin{titlepage}
    \begin{center}
        \vspace*{1cm}
 
        \Huge
        \textbf{Experiment 4: Series and Parallel Circuits}
 
        \vspace{0.5cm}
        \LARGE
        PHYS LAB 2240
 
        \vspace{1.5cm}
 
        \textbf{Simon Xiang}
 
        \vfill
  
        \vspace{0.8cm}
 
        \Large
        Physics Lab Section 502\\
        University of North Texas\\
        \today
 
    \end{center}
\end{titlepage}

\section{Abstract}
The objective of this experiment was to show the inner workings of circuits in parallel and series and calculate
their respective equivalent resistances. To do this, four circuits were set up, giving the experimenter hands on
circuit experience in the laboratory. Another objective of this experiment was to show how lightbulbs behaved when
connected in series and in parallel.

Our results were extremely precise and reflected the accuracy of the theoretical formula for deriving $R_{eq}$ as
opposed to the experimental method. The average percent difference between the two $R_{eq}$ values across all four 
circuit diagrams was $1.29\%$, showing that our experimental results are consistent with Ohm's Law due to the extremely
low percent difference. This experiment's significance is that it demonstrates one of the most fundamental laws in physics (Ohm's Law), 
shows how to calculate resistance of circuits in series and parallel, and gives hands on experience in the lab, with
future applications including setting up circuits and anything involving resistors.
\section{Introduction}
Several fundamental concepts are demonstrated within this experiment. The first is Ohm's Law, mathematically given 
by the formula
\begin{equation} \label{eq:1}
    V = IR, 
\end{equation}
where $V$ is the voltage through a given conductor, $I$ is the current through such conductor, and $R$ being defined as 
the ratio $V/I$. The voltage $V$ is measured in volts ($1 V = 1 \frac{J}{C}$), the current $I$ is measured in amperes ($1 A = 1 \frac{C}{s}$),
and the resistance $R$ is measured in ohms ($1 \Omega = 1 \frac{V}{A} = 1 \frac{J \cdot s}{C^2}$). 

In the case of a device that obeys Ohm's Law, such resistance $R$ is constant, and such devices
are said to be Ohmic. Resistors are known to be Ohmic. We define an \textbf{equivalent resistor} as a resistor that can replace a more complex circuit,
i.e., produce the same amount of current when the same amount of voltage is applied. Refer to Figure 1 for the circuit diagram
of two resistors in series:
\begin{figure}[ht]
    \begin{minipage}[b]{0.45\linewidth}
        \begin{circuitikz} \draw
            (0,0) to[short, l^=$I \bigg\uparrow $,-*] (0,4)
            (0,4) to[R,l^=$R_1$, -] (2,4)
            (2,4) to[R,l^=$R_2$, -*] (4,4)
            (4,4) to[short, l^=$\bigg\downarrow I$] (4,0)
            (4,0) to[C,l_=$V$] (0,0);
        \end{circuitikz}
    \caption{Resistors in series.}
    \label{fig:1}
\end{minipage}
\quad
\begin{minipage}[b]{0.45\linewidth}
    \begin{circuitikz} \draw
        (0,0) to[short, l^=$I \bigg\uparrow $,-*] (0,4)
        (0,4) to[R,l^=$R_{eq}$, -*] (4,4)
        (4,4) to[short, l^=$\bigg\downarrow I$] (4,0)
        (4,0) to[C,l_=$V$] (0,0);
    \end{circuitikz}
    \caption{Equivalent resistor.}
    \label{fig:2}
\end{minipage}
\end{figure}
We have by the conservation of energy 
\begin{equation*}
    V = V_1 + V_2
\end{equation*}
for two resistors in series. Consider Ohm's Law: since $V_1 = IR_1$, $V_2 = IR_2$, we have
\begin{equation*}
        V = IR_1 + IR_2=I(R_1 + R_2).
\end{equation*}
So
\begin{equation*}
    R_1+R_2 = \frac{V}{I},
\end{equation*}
and we conclude that 
\begin{equation} \label{eq:2}
    R_{eq} = R_1 + R_2.     
\end{equation}
This can be represented by the following circuit diagram anagalous to Figure 1, shown in Figure 2.
In the case that the resistors are in parallel, we have
\begin{equation*}
    V = V_1 = V_2,
\end{equation*}
since each resistor recieves all the current. Refer to the circuit diagram given by Figure 3 for more information. By Ohm's Law and the conversation of charge, we have
\begin{equation*}
    I = I_1 + I_2 = \frac{V}{R_1} + \frac{V}{R_2},
\end{equation*}
therefore 
\begin{equation*}
    \frac{1}{R_1} + \frac{1}{R_2} = \frac{I}{V},
\end{equation*}
so we can safely conclude that
\begin{equation*}
    \frac{1}{R_{eq}} = \frac{1}{R_1} + \frac{1}{R_2}.
\end{equation*}
Some simple algebra yields the equation
\begin{equation} \label{eq:3}
    R_{eq} = \frac{R_1R_2}{R_1 + R_2}.
\end{equation}
See Figure 4 for an example of what the equivalent resistor for resistors in parallel look like.
\begin{figure}[H]
    \begin{minipage}[b]{0.45\linewidth}
        \begin{circuitikz} \draw
            (0,0) to[short, l^=$I \Big\uparrow $] (0,2)
            (0,2) to[short, l^=$I_1 \Big\uparrow $] (0,4)
            (0,4) to[short] (1,4)
            (1,4) to[R,l^=$R_1$] (4,4)
            (4,4) to[short] (5,4)
            (5,4) to[short] (5,0)
            (5,0) to[C,l_=$V$] (0,0);
            \draw
            (0,2) to[short,l^=$\xrightarrow{I_2}$] (1,2)
            (1,2) to[R,l^=$R_2$] (4,2)
            (4,2) to[short] (5,2);
        \end{circuitikz}
        \caption{Resistors in parallel.}
        \label{fig:3}
\end{minipage}
\quad
\begin{minipage}[b]{0.45\linewidth}
    \begin{circuitikz} \draw
        (0,0) to[short, l^=$I \Big\uparrow $] (0,3)
        (0,3) to[R,l^=$R_{eq}$] (5,3)
        (5,3) to[short] (5,0)
        (5,0) to[C,l_=$V$] (0,0);
    \end{circuitikz}
    \caption{Equivalent resistor for resistors in parallel.}
    \label{fig:4}
\end{minipage}
\end{figure}
\section{Apparatus}
Apparatus used include an AC/DC Electronics lab, a Digital Storage Oscilloscope and its probes, a DMM (Digital Multimeter),
a DC Power Supply, six resistors, and three lightbulbs.

The six resistor bands came in two sets of three resistance levels, being $330 \Omega, 560 \Omega$, and 
$1000 \Omega$ respectively. The DMM was crucial in experimentally determining the resistance of such resistors
and verifying whether they fell within the percent error indicated by the color of the last band of the resistor 
($\pm 5\%$ for this experiment). The DMM also measured the current of the entire circuit, powered by the 
DC power supply itself. Finally, current was provided to the lightbulbs to demonstrate how they changed (or didn't) in brightness as current varied
for the last portion of the experiment.
\section{Experimental Procedure}
\subsection*{Part A: Series and Parallel Resistor Circuits}
Preceding the experiment, the DMM was used to measure the resistance values for the six resistors and verify that they indeed
fell within the given percent error indicated by the color of the last band on the resistors. 

Consider the following four circuit diagrams on the page below- we will
refer to them heavily when describing the experimental procedure. 
\begin{figure}[H]
    \begin{minipage}[b]{0.45\linewidth}
        \begin{circuitikz} \draw
            (0,0) to[R,l^=\small$R_{2} \equal 560\Omega$, o-] (2.5,0)
            (2.5,0) to[R,l^=\small$R_{1} \equal 330\Omega$, -o] (5,0);
        \end{circuitikz}
    \caption*{Circuit Diagram 1: Series}
    \label{fig:cd1}
\end{minipage}
\quad
\begin{minipage}[b]{0.45\linewidth}
    \begin{circuitikz} \draw
        (0,0) to[R,l_=\small$R_{2} \equal 560\Omega$] (0,2.5)
        (0,2.5) to[short, -o] (5,2.5)
        (0,0) to[short, -o] (5,-0)
        (2.5,0) to[R,l_=\small$R_{1} \equal 330\Omega$] (2.5,2.5);
    \end{circuitikz}
    \caption*{Circuit Diagram 2: Parallel}
    \label{fig:cd2}
\end{minipage}
\end{figure}
\begin{figure}[H]
    \begin{minipage}[b]{0.45\linewidth}
        \begin{circuitikz} \draw
            (0,0) to[R,l_=\small$R_{2} \equal 560\Omega$] (0,2.5)
            (0,2.5) to[short, -o] (5,2.5)
            (2.5,0) to[R,l_=\small$R_{1} \equal 330\Omega$] (2.5,2.5)
            (0,0) to[short] (2.5,0)
            (2.5,0) to[R,l^=\small$R_{3} \equal 1000\Omega$, -o] (5,0);
        \end{circuitikz}
    \caption*{Circuit Diagram 3: Simple Series Parallel}
    \label{fig:cd3}
\end{minipage}
\quad
\begin{minipage}[b]{0.45\linewidth}
    \begin{circuitikz} \draw
        (0,0) to[short, -o] (6,0)
        (0,0) to[R,l_=\footnotesize$R_{6} \equal 330\Omega$] (0,2.5)
        (2,0) to[R,l_=\footnotesize$R_{5} \equal 560\Omega$] (2,2.5)
        (4,0) to[R,l_=\footnotesize$R_{2} \equal 560\Omega$] (4,2.5)
        (0,2.5) to[R,l^=\footnotesize$R_{4} \equal 1000\Omega$] (2,2.5)
        (2,2.5) to[R,l^=\footnotesize$R_{1} \equal 330\Omega$] (4,2.5)
        (4,2.5) to[R,l^=\footnotesize$R_{3} \equal 1000\Omega$, -o] (6,2.5);
    \end{circuitikz}
    \caption*{Circuit Diagram 4: Complex Series Parallel}
    \label{fig:cd4}
\end{minipage}
\end{figure}

    The experiment began by setting up the circuit according to Circuit Diagram 1 above. Following,
    the DMM was inserted \textbf{in series}, since the circuit was set up in series. After adjusting the 
    DMM to the proper units ($mA$), the DC power supply was activated and a quick check was done to see that
    both the voltage and current were set to zero by turning their respective knobs to zero on the power supply.
    Then, the power supply and DMM were connected to the circuit, and the voltage on the DC Power Supply was set to $15.00 V$.
    Using the current knob, the current was \textbf{slowly} increased with small additions (about $0.010 A$) until
    a voltage of $15.00 V$ was reached. From this, the DMM was used to measure the current, and the power supply was turned off.

In terms of calculations, the \textit{theoretical} $R_{eq}$ was calculated by referring to Equation \ref{eq:2} since the circuit
was set up in series. Particular care was taken in making sure that the measured values were used for the resistance
as opposed to the given base values. The \textit{measured} $R_{eq}$ was calculated by Ohm's law (Equation \ref{eq:1}).
Then, the percent difference between the two values was calculated using the handy formula
\begin{equation} \label{eq:4}
    \text{Percent Difference} = \frac{\mid \!theoretical - measured \mid}{\frac{\mid theoretical + measured \mid}{2}} \cdot 100\%.
\end{equation}
Now the circuit was disassembled and re-setup in the pattern of Circuit Diagram 2, in parallel. Everything following the step 
when the voltage on the DC Power Supply was set to $15.00 V$ was repeated for the newly setup circuit. Something particular to note 
was that the theoretical $R_{eq}$ was calculated using Equation \ref{eq:3} as opposed to Equation \ref{eq:2} since
the circuit was set up in parallel. This pattern followed for Circuit Diagrams 3 and 4. For Circuit Diagram 3, 
the theoretical $R_{eq}$ was calculated by first considering $R_1$ and $R_2$ in parallel, then considering the resultant
$R_{eq}$ in series with $R_3$. This is shown in Equation \ref{eq:5} in the Calcuations section below. For Circuit Diagram 4, $R_4$ and $R_6$ were considered in series, then the resultant equivalent resistor
was considered in parallel with $R_5$. Then, such resultant equivalent resistor was considered in series with $R_1$, which was then
considered in parallel with $R_2$, which was then finally considered in series with $R_3$ to complete the calculation of the 
equivalent resistor of the entire circuit, represented by Equation \ref{eq:6}.

A detailed mathematical representation of these calculations can be found in the Calculations section of this report.
\subsection*{Part B: Series and Parallel Lightbulb Circuits}
Unfortunately I am unaware of the capabilities of \LaTeX{} to generate lightbulb figures, so the terms "series", "parallel", and "series parallel"
will have to do when describing the setup of the circuit board. We begin by connecting lightbulb A in series to the power supply
and setting the voltage to $2.00V$. Once again, we slowly increase the current to reach $2.00V$ (we expect to end somewhere about $0.260A$).
Using the values \textit{displayed on the power supply}, we determine the resistance of A by Ohm's Law (Equation \ref{eq:1}).
Rinse and repeat for lightbulbs B and C. After determining the resistance of all three bulbs, set up the lightbulbs A and B in series and determine
the resistance. Then, set up lightbulbs A and B in parallel and determine the resistance. Finally, set up all three lightbulbs in the
series parallel setup and determine the resistance. 


Some issues included a resistor breaking in half halfway through the experiment, and the peculiar case of the lightbulbs
lighting up in the wrong order even with a correct setup as verified by our lab TA's. The first problem was resolved by grabbing 
a new resistor from another table since we hadn't used that particular resistor yet at that point in time. The second problem was never resolved.

\section{Data}
\textbf{Table 1}: Data obtained from measuring the resistance values of the six given resistors with the DMM.
These values were used to calculate the theoretical resistance in junction with the derived equations for $R_{eq}$.
\renewcommand{\arraystretch}{1.5}
\renewcommand{\tabcolsep}{0.2cm}
\begin{center}
    \begin{tabular}{|c|c|c|c|c|c|}
        \hline
        $R_1 = 330\Omega$ & $R_2 = 560\Omega$ & $R_3 = 1000\Omega$ & $R_4 = 1000\Omega$ & $R_5 = 560\Omega$ & $R_6 = 330\Omega$\\ 
        \hline
        $325.5\Omega$ & $544\Omega$ & $978\Omega$ & $980\Omega$ & $552\Omega$ & $324.3\Omega$ \\
        \hline
    \end{tabular}
\end{center}
\vspace{2cm}
\textbf{Table 2}: Data obtained from experimentally determing the value(s) of $R_{eq}$ for each circuit diagram.
Here, the data from Table 1 was used to calculate the theoretical $R_{eq}$, the measured current and $R_{eq}$ were experimentally determined
in junction with Ohm's Law (with $V$ being a constant $15V$, see Equation \ref{eq:1}), and the percent difference was calculated with
Equation \ref{eq:4}. These calculations are all shown in the manual calculation section of this lab report.
\begin{center}
    \begin{tabular}{|m{2cm}|m{2cm}|m{2cm}|m{2cm}|m{2cm}|}
        \hline
        Circuit \qquad Diagram & Theoretical $R_{eq} (\Omega)$ & Measured Current $(mA)$ & Measured $R_{eq} (\Omega) $ & Percent \quad Difference\\ 
        \hline
        1 & 869.5 & 17.1 & 877.2 & 0.88\%\\ 
        \hline
        2 & 203.65 & 71.2 & 210.7 & 3.39\%\\ 
        \hline
        3 & 1181.65 & 12.6 & 1190.5 & 0.74\%\\ 
        \hline
        4 & 1291.28 & 11.6 & 1293.1 & 0.14\%\\ 
        \hline
    \end{tabular}
\end{center}
\vspace{0.2cm}
\textbf{Table 3:} Finally we reach the table of data for the lightbulb portion of this experiment. These values are entirely
experimental and were all measured from the reading off of the power supply and letting the voltage $V$ equal two volts and solving
for resistance by Ohm's Law. 
\begin{center}
    \begin{tabular}{|c|c|}
    \hline
    Bulb System & Measured Resistance $(\Omega)$ \\
    \hline
    A & 7.38 \\
    \hline
    B & 8.13 \\
    \hline
    C & 15.873 \\
    \hline
    A\&B Series & 10.582 \\
    \hline
    A\&B Parallel & 3.914 \\
    \hline
    Series Parallel & 8.811 \\
    \hline
    \end{tabular}
\end{center}
\section{Calculations}
We begin this section by deriving the precise mathematical formulas for the calculation of the respective $R_{eq}$'s for each circuit diagram.
Consider Circuit Diagram 1 and 2. Clearly the resistors are simply in series or parallel, so a direct application
of Equation \ref{eq:2} and \ref{eq:3} will suffice in calculating the $R_{eq}$ for these circuit diagrams. Let us turn our attention to 
Circuit Diagram 3, once again displayed for clarity:
\begin{figure}[H]
    \begin{center}
        \begin{circuitikz} \draw
            (0,0) to[R,l_=\small$R_{2} \equal 560\Omega$] (0,2.5)
            (0,2.5) to[short, -o] (5,2.5)
            (2.5,0) to[R,l_=\small$R_{1} \equal 330\Omega$] (2.5,2.5)
            (0,0) to[short] (2.5,0)
            (2.5,0) to[R,l^=\small$R_{3} \equal 1000\Omega$, -o] (5,0);
        \end{circuitikz}
    \caption*{Circuit Diagram 3: Simple Series Parallel}
    \label{fig:cd3}
    \end{center}
    \end{figure}
Now clearly the resistors $R_1$ and $R_2$ are in parallel, so we apply Equation \ref{eq:3} to obtain
\begin{equation*}
    R_{eq(1)}= \frac{R_1R_2}{R_1 + R_2}.
\end{equation*}
From here, the resultant equivalent resistor is in series with the resistor $R_3$, so we apply Equation \ref{eq:2}
to obtain
\begin{equation} \label{eq:5}
    R_{eq} = R_{eq(1)} + R_3 = \frac{R_1R_2}{R_1 + R_2} + R_3
\end{equation}
to obtain the $R_{eq}$ for the entirety of the system in Circuit Diagram 3. Now let us turn our attention once more to 
Circuit Diagram 4 below:
\begin{figure}[H]
    \begin{center}
    \begin{circuitikz} \draw
        (0,0) to[short, -o] (6,0)
        (0,0) to[R,l_=\footnotesize$R_{6} \equal 330\Omega$] (0,2.5)
        (2,0) to[R,l_=\footnotesize$R_{5} \equal 560\Omega$] (2,2.5)
        (4,0) to[R,l_=\footnotesize$R_{2} \equal 560\Omega$] (4,2.5)
        (0,2.5) to[R,l^=\footnotesize$R_{4} \equal 1000\Omega$] (2,2.5)
        (2,2.5) to[R,l^=\footnotesize$R_{1} \equal 330\Omega$] (4,2.5)
        (4,2.5) to[R,l^=\footnotesize$R_{3} \equal 1000\Omega$, -o] (6,2.5);
    \end{circuitikz}
    \caption*{Circuit Diagram 4: Complex Series Parallel}
    \label{fig:cd4}
\end{center}
\end{figure}
This indeed is a very complex setup. The resulting formula will be quite nasty, but do not fear, as solving for $R_{eq}$ iteratively by 
plugging in values obtained at each step is much simpler. However, we will still write an expression for the formula in terms of the 
given quantities for the sake of mathematical purity. We have resistors $R_4$ and $R_6$ in series, so    
\begin{equation*}
    R_{eq(1)} = R_4+R_6.
\end{equation*}
Then we take $R_{eq(1)}$ in parallel with $R_5$ and we get 
\begin{equation*}
    R_{eq(2)} = \frac{R_{eq(1)}R_5}{R_{eq(1)}+R_5} = \frac{(R_4+R_6)R_5}{R_4 + R_5 + R_6}.
\end{equation*}
We repeat, taking $R_{eq(2)}$ in series with $R_1$, yielding
\begin{equation*}
    R_{eq(3)} = R_{eq(2)} + R_1 = \frac{(R_4+R_6)R_5}{R_4 + R_5 + R_6} + R_1, 
\end{equation*}
and so the resultant equivalent resistor is in parallel with $R_2$, so yet another application of Equation \ref{eq:3} yields
\begin{equation*}
    R_{eq(4)} = \frac{R_{eq(3)}R_2}{R_{eq(3)}+R_2} = \frac{\big(\frac{(R_4+R_6)R_5}{R_4 + R_5 + R_6} + R_1\big)R_2}{\frac{(R_4+R_6)R_5}{R_4 + R_5 + R_6} + R_1 + R_2}.
\end{equation*}
Finally, we take $R_{eq(4)}$ in series with $R_3$ to obtain the final expression
\begin{equation} \label{eq:6}
    R_{eq} = R_{eq(4)} + R_3 = \frac{\big(\frac{(R_4+R_6)R_5}{R_4 + R_5 + R_6} + R_1\big)R_2}{\frac{(R_4+R_6)R_5}{R_4 + R_5 + R_6} + R_1 + R_2} + R_3.
\end{equation}
Our fruitless toil comes to a conclusion.
\subsection*{Manual Calculations}
Now we begin the menial task of plugging in numbers into these formulas to get theoretical values for $R_{eq}$.
We use the data from Table 1, which we once again show for clarity:
\begin{center}
    \begin{tabular}{|c|c|c|c|c|c|}
        \hline
        $R_1 = 330\Omega$ & $R_2 = 560\Omega$ & $R_3 = 1000\Omega$ & $R_4 = 1000\Omega$ & $R_5 = 560\Omega$ & $R_6 = 330\Omega$\\ 
        \hline
        $325.5\Omega$ & $544\Omega$ & $978\Omega$ & $980\Omega$ & $552\Omega$ & $324.3\Omega$ \\
        \hline
    \end{tabular}
\end{center}
\vspace{0.5cm}
Now calculating $R_{eq}$ for Circuit Diagrams 1 and 2 is trivial: take Equations \ref{eq:2} and \ref{eq:3} respectively and simply plug in the data from the table.
For Circuit Diagram 1, we have 
\begin{equation*}
    R_{eq} = 325.5 + 544 = 869.5,
\end{equation*}
and for Circuit Diagram 2 we have
\begin{equation*}
    R_{eq} = \frac{325.5 \cdot 544}{325.5 + 544} = 203.65.
\end{equation*}
Plug in the values of Table 1 into Equation \ref{eq:5} to obtain
\begin{equation*}
    R_{eq} = \frac{325.5 \cdot 544}{325.5 + 544} + 978 = 1181.65
\end{equation*}
for Circuit Diagram 3. For Circuit Diagram 4, we use the obtrusively long Equation \ref{eq:6} to get
\begin{equation*}
    R_{eq} = \frac{\frac{(980+324.3)552}{980+552+324.3}+325.5}{\frac{(980+324.3)552}{980+552+324.3}+325.5+544}+978=1291.28.
\end{equation*}

Now that we have finished calculating the theoretical $R_{eq}$'s, we turn our attention to manually calculating the measured $R_{eq}$'s
by Ohm's Law. Ohm's Law (or Equation \ref{eq:1}) states that $V=IR$, or $R=\frac{V}{I}$. Here the voltage remains at a constant $15 V$, so for Circuit Diagram 1,
$R_{eq} = \frac{15V}{17.1mA} = 877.2\Omega$, $R_{eq} = \frac{15V}{71.2mA} = 210.7\Omega$ for Circuit Diagram 2, 
$R_{eq} = \frac{15V}{12.6mA} = 1190.48\Omega$ for Circuit Diagram 3, and finally $R_{eq} = \frac{15V}{11.6mA} = 1293.1\Omega$ for
Circuit Diagram 4.

Calculating percent differences is also a tedious task: apply Equation \ref{eq:4} four times to the data above.
For Circuit Diagram 1, we have
\begin{equation*}
    \text{Percent Difference} = \frac{\mid \! 869.5-877.2  \mid}{\frac{\mid 869.5+877.2 \mid}{2}} \cdot 100\% = 0.88\%,
\end{equation*}
for Circuit Diagram 2
\begin{equation*}
    \text{Percent Difference} = \frac{\mid \!203.65 - 210.7 \mid}{\frac{\mid 203.65 + 210.7 \mid}{2}} \cdot 100\% = 3.39\%,
\end{equation*}
for Circuit Diagram 3
\begin{equation*}
    \text{Percent Difference} = \frac{\mid \! 1181.65-1190.48  \mid}{\frac{\mid  1181.65+1190.48  \mid}{2}} \cdot 100\% = 0.74\%,
\end{equation*}
and finally for Circuit Diagram 4
\begin{equation*}
    \text{Percent Difference} = \frac{\mid \! 1291.28-1293.1  \mid}{\frac{\mid  1291.28+1293.1  \mid}{2}} \cdot 100\% = 0.14\%,
\end{equation*}
which concludes the calculations portion of this lab report.
\section{Discussion of Results and Error Analysis}
There are several possible sources of error of this experiment. The first is the inherent internal resistance of the wires and 
resistors themselves- although theoretically they match up with the Ohm reading indicated in the manual, there will always be a tolerance error
of about $(\pm 5\%)$ in the resistors themselves. This is indicated by the color of the last band of the resistor, since in this experiment they
were all gold, the error was within the $(\pm 5\%)$ threshold. Now to further minimize error we used the measured values of the resistance of the resistors to calculate 
the theoretical $R_{eq}$'s, however, further sources of error within the experiment persist. One of them is the equipment- things like how accurate
the DMM is in measuring current, how close the voltage the power supply puts out is to $15V$, the quality of the circuit board, how strong the lightbuls are, and so on.
In our particular experiment, a faulty circuit board led to confusion in Part B of the experiment- the lightbulb not receiving the current first shone much brighter than the
one that was supposed to be brighter. 

Furthermore, the ambient tempurature of the room also affects current and therefore voltage and resistance. Conditions will never be 
$100\%$ ideal, and therefore there will always be error within the experiment. However, our results were relatively good, with extremely low percent differences,
the lowest being $0.14\%$ and the highest being $3.39\%$, which is still incredibly low (and much much higher than the other percent differences too). 
The average percent difference was $1.29\%$, showing that even with the many sources of error listed above the experiment prove to be quite accurate and true to the 
theoretical results. 
\section{Conclusion}
It appears as if it is time for this lab report to reach a conclusion. We have considered several equations of relative importance, such
as Ohm's Law and resistors in series and parallel, and learned about equivalent resistors and how they can be used to simply resistor circuit
calculations. We verified that the resistance of resistors in series added additively, while 
resistors in parallel added reciprocally experimentally. We also studied more complex circuit diagrams and dove into how to calculated equivalent resistors for such systems.
There was also a section demonstrating the effect of current on the strength of various lightbulbs.


Based off of the extremely low percent errors, our results seem to completely support the theory behind the calcution of $R_{eq}$ for
various resistor setups. Such errors can possibly be eliminated or reduced with better equipment, ideal room conditions, and more precise resistors. 
However not all of these options are available to an undergraduate physics lab section (some may not even be available to research labs), and yet our measured results were
still quite spot on with the theoretical values calculated. This means that the method used is precise enough to obtain such a low percent error (average of $1.29\%$), and therefore
sufficiently accurate. We conclude that resistors do indeed follow Ohm's Law, equivalent resistors for resistors in series and parallel are indeed calculated the way that we derived in the introduction,
and that modern physics as we know it is indeed somewhat accurate.
\end{document}

\documentclass{article}
\usepackage{amsmath, amssymb, IEEEtrantools}
\usepackage{amsthm}
%\usepackage{theoremref}
\usepackage{amssymb}
\usepackage{graphicx}
\usepackage{array}
\usepackage{circuitikz}
\usepackage{footnote}
\usepackage{caption}
\usepackage{hyperref}
\usepackage{pgflibraryarrows}
\usepackage [english]{babel}
\usepackage [autostyle, english = american]{csquotes}
\MakeOuterQuote{"}
\usetikzlibrary{arrows}
\hypersetup{
    colorlinks=false,
    linkcolor=blue,
    filecolor=magenta,      
    urlcolor=blue,
}
\newcommand*{\equal}{=}
\renewcommand\qedsymbol{$\boxtimes$}

\title{Representation of Mathematical Expressions in \LaTeX}
\author{Simon Xiang}
\date{\today}
\newtheorem{theorem}{Theorem}

\begin{document}
\begin{titlepage}
    \begin{center}
        \vspace*{1cm}
 
        \Huge
        \textbf{Experiment 6: Kirchhoff's Circuit Laws}
 
        \vspace{0.5cm}
        \LARGE
        PHYS LAB 2240
 
        \vspace{1.5cm}
 
        \textbf{Simon Xiang}
 
        \vfill
  
        \vspace{0.8cm}
 
        \Large
        Physics Lab Section 502\\
        University of North Texas\\
        \today
 
    \end{center}
\end{titlepage}

\section{Abstract}
The objective for this particular experiment was to demonstrate the use of Kirchhoff's Laws in solving electrical circuits.
In this case, a "Y" circuit was analyzed using Kirchhoff's Laws. As is with every physics lab, another objective was to familiarize oneself
with using laboratory equipment like the DMM to measure current and voltages, and the use of the DC power supply in setting up circuits. 

The results matched our hypothesis: indeed Kirchhoff's Laws proved an invaluable tool in calculating the theoretical currents, and the percent differences
between the measured and theoretical results further support our case. The average percent difference between the theoretical and
measured values was $1.74\%$, showing consistency of the measured results with Kirchhoff's Laws. The significance of this experiment lies in 
the fact that it serves as an example to one of the fundamental (sets of) laws in Electromagnetism, teaches students valuable skills like plugging numbers into an RREF calculator,
and allows for hands on experience in the lab, something that is quite important I believe. 
\section{Introduction}
We begin by stating by stating Kirchhoff's Laws.
\begin{theorem}[Junction Law]\label{junction}
Let $A$ be a junction point of an electrical circuit. Then the sum of the current flowing in and out of $A$
must equal zero, that is,
\begin{equation*}
    \sum_{k=1}^n I_{k} = 0,
\end{equation*}
where $n$ denotes the number of branches of currents that flow in or out of the node.
\end{theorem}
\begin{proof}
    Follows from the conservation of energy.
\end{proof}
\begin{theorem}[Loop Law]\label{loop}
    Let $V_k$ denote voltages around a closed loop. Then the sum of the voltages must equal zero for $n$ the number of voltages measured, that is,
\begin{equation*}
    \sum_{k=1}^n V_{k} = 0.
\end{equation*}
\end{theorem}
\begin{proof}
    Left to the reader as an exercise.\footnote{Refer to this link for more information: \url{https://www.feynmanlectures.caltech.edu/II_22.html}}
\end{proof}
The sample circuit as shown in the lab manual has been omitted since circuit diagrams take a long time to generate. You can consider the visualization of the
sample circuit as yet another exercise left to the reader. Now we move on to analyze the particular circuit setup that this experiment concerns itself with. 
Refer to Figure \ref{fig:1} and Figure \ref{fig:2} to see the "Y" circuit setup and the Voltage Divider from which we will calculate $V_f$ respectively.

\begin{figure}[!ht]
    \begin{minipage}{.6\textwidth}
    \centering
    \begin{circuitikz} \draw
        (0,0) to[C, l^=$V_f$, -o] (0,6)
        (0,6) to[R, l=$R_1 \equal 100 \Omega$, i_>=$I_1$] (3,3)
        (6,6) to[R, l_=$R_2 \equal 330 \Omega$, i^>=$I_2$, o-] (3,3)
        (6,6) to[C, l=$V_0$, o-] (6,0)
        (3,3) to[R, l=$R_3 \equal 100 \Omega$, i^>=$I_3$, *-] (3,0)
        (3,0) to (3,-0.2) node[ground]{}
        (3,3) node[label={[font=\footnotesize]above:A}]{} to (3,3) 
        (0,0) to (6,0);
        \draw[thin, ->, >=triangle 45] (1.5,2.6)node{$1$}  ++(-180:0.5) arc (-180:180:0.5);
        \draw[thin, ->, >=triangle 45] (4.5,2.6)node{$2$}  ++(-180:0.5) arc (-180:180:0.5);
    \end{circuitikz}
    \caption{"Y" Circuit.}
    \label{fig:1}
    \end{minipage}
    \begin{minipage}{.4\textwidth}
    \centering
    \begin{circuitikz} \draw
        (2,6) to[R, l=$R_4 \equal 330\Omega$, o-] (2,3)
        (2,3) to[R, l=$R_5 \equal 1000\Omega$] (2,0)
        (2,0) -- (2,-0.2) node[ground]{};
        \draw[-latex] (2,3) -- (3,3) node[anchor=west]{$V_f$};
    \end{circuitikz}
    \caption{Voltage Divider.}
    \label{fig:2}
    \end{minipage}
\end{figure}
From here a simple application of Kirchhoff's Laws will generate a complete set of equations that can be solved.
This result is non-trivial and is also of mathematical interest, a general idea for the proof comes from the mapping of a circuit 
to a graph and from a graph to a matrix whos rows are vertices and columns are edges. Then after defining some vector spaces for the currents and voltages,
our result follows from some neat linear algebra. For the interested reader, a link to the full paper containing the proof can be found below. \footnote{An archive of the paper can be found by clicking \href{http://web.archive.org/web/20150122071322/http://www2.math.uu.se/~takis/L/Circuits/2000/handouts/graphsandckts/graphsandckts.pdf}{\textbf{here}}.}
Now in the case of our particular setup, a simple application of Theorem
\ref{junction} to node A in the "Y" circuit yields the equation
\begin{equation}\label{eq:1}
    I_1 + I_2 - I_3 = 0,
\end{equation}
definiting positive current as flowing into the node and negative current as flowing out of the node.
We apply Theorem \ref{loop} to the closed loops $1$ and $2$ respectively to get the equations
\begin{equation}\label{eq:2}
    R_1 I_1 + R_3 I_3 - V_f = 0
\end{equation}
\begin{equation}\label{eq:3}
    -R_2 I_2 - R_3 I_3 + V_0 = 0.
\end{equation}
Note that in this case, we define the voltage as negative when it flows from a higher voltage to a lower one.
A simple application of Ohm's Law yields the following equations for calculating the experimental values of $I_1$ and $I_2$, given by
\begin{equation}\label{eq:4}
    I_1 = \frac{V_f - V_B}{R_1}
\end{equation}
and 
\begin{equation} \label{eq:5}
    I_2 = \frac{V_0 - V_B}{R_2},
\end{equation}
respectively. It can be seen that $I_1$ and $I_2$ are calculated by measuring the voltage drops between $V_f$ and $V_B$, and $V_0$ and $V_B$, respectively.

One final note: the "$A$" in Figure \ref{fig:1} refers to the node at which $I_1$ and $I_2$ flow in, and $I_3$ flows out, not to be confused with the point $B$ at the same location
which is used to measured the voltages above and set up the actual circuit in lab. Notice that $B$ is not shown, along with its friends
$A,C,D,E,F,$ and $G$. This is a purely stylistic choice, it has nothing to do with the fact that I forgot to add them in when generating the figure. Please 
use your imagination to visualize them. Thank you in advance. 
\section{Apparatus}
The apparatus used included an AC/DC Electronics Lab, a DC Power Supply, a DMM, a set of Resistors, and a set of 
eight Short Patch Cords.

The resistors came in a set of five, with resistances of $100\Omega$, $330\Omega$, $100\Omega$, $330\Omega$, and $1000\Omega$ respectively, corresponding to the resistor's respective index number.
The electronics lab was used to set up the entire experiment, and the DC Power supply was used to provide power so we could measure the voltages and determine the currents.
The DMM was used to measure the resistance of the resistors and the voltages of the circuit. 
Quite frankly, I forgot what a short patch cord was. I think they were the things that connected
the points to the power supply. Yeah, let's go with that. 

\section{Experimental Procedure}
\subsection*{Part A: DC Analyis of Kirchhoff's Laws}
Before the experiment began, the DMM was used to measure the resistance of the resistors. 
The aforementioned difficult of generating circuit diagrams unfortunately results in a lack of circuit to emulate. An effort was done 
to construct the circuit according the lab manual, shaped like a "T" with points $A, B,$ and $C$ across the upper portion of the "T", and
the point $D$ and the base of the "T". $R_1$ constituted $\overline{AB}$, $R_2$ constituted $\overline{BC}$, and $R_3$ constituted $\overline{BD}$.
Similarly, the voltage divider was constructed similar to the shape of a backwards "L" (L for my grade on the last physics test), with points $G, F,$ and $E$ 
at the left most point, connector point, and topmost point respectively. $R_5$ constituted $\overline{GF}$ and $R_4$ constituted $\overline{FE}$.

Now we connect the voltage divider to the "Y" ("T"?) circuit. The two were connected by connection the following points:
$C$ to $E$, $D$ to $G$, and $A$ to $F$, using connecting wires. Using banana plugs and alligator clips, the DC Power Supply was then connected
to point $E$ by the positive terminal and point $G$ by the ground terminal. 
This completes the setup of the circuits.

The power supply was then activated and the voltage set to $15.00 V$. Then, a miniscule current was provided to keep the voltage,
about $0.060 A$. The DMM was used to measure the voltage $V_0$ by touching the black lead to some ground connection, and the red lead to point $E$.
Then, the DMM was used to measure the voltages $V_f$ and $V_B$ by keeping the black lead stationary from the previous step, 
and touching the red lead to their respective points on Figure \ref{fig:1}.
The DC Power supply was then shut off.

$I_3$ was then measured in series with the DMM via the resistor $R_3$. The wire that connected points 
$D$ and $G$ was removed and the DMM was inserted in its place by touching the black lead to point $G$ and the read lead to point $D$. It was noted that the DMM was set to $mA$. This gave the current going through $R_3$. 
The DC Power supply was then, once again, shut off. A similar process was repeated to measure $I_1$ and $I_2$. For $I_1$, the black lead was connected to point $A$ and the red lead to point $F$.
For $I_2$, the black lead was connected to point $C$ and the red lead to point $E$. Note that Equations
\ref{eq:4} and \ref{eq:5} could have been used as well to calculate $I_1$ and $I_2$, respectively. Turn off the power supply for the final time.

Finally, the system of linear equations given by Equations \ref{eq:1}, \ref{eq:2}, and \ref{eq:3} were solved using methods from linear algebra to yield the theoretical values, and
the percent differences between the theoretical and measured values were calculated using the handy formula
\begin{equation}\label{eq:6}
    \text{Percent Difference} = \frac{\mid \!theoretical - measured \mid}{\frac{\mid theoretical + measured \mid}{2}} \cdot 100\%.
\end{equation}
See the Calculations section of this report for more information. 
\subsection*{Part B: There is none}
Why is there a "Procedure A" in the lab manual if there is no "Procedure B" to follow it?
\section{Data}
\renewcommand{\arraystretch}{1.5}
\renewcommand{\tabcolsep}{0.2cm}
\textbf{Table 1:} Data obtained from measuring the resistance of the resistors with the DMM. 
\begin{center}
    \begin{tabular}{|c|c|c|c|c|}
        \hline
        $R_1 = 100\Omega$ & $R_2 = 330\Omega$ & $R_3 = 100\Omega$ & $R_4 = 330\Omega$ & $R_5 = 1000\Omega$\\ 
        \hline
        $99.6\Omega$ & $326\Omega$ & $99.8\Omega$ & $326\Omega$ & $985\Omega$\\
        \hline
    \end{tabular}
\end{center}
\vspace{2cm}
\textbf{Table 2:} Data obtained from measuring the voltages and currents experimentally.
\begin{center}
    \begin{tabular}{|c|c|c|c|c|c|}
        \hline
        $V_0$ & $V_B$ & $V_f$ & $I_1$ & $I_2$ & $I_3$\\ 
        \hline
        $14.97 V$ & $4.91 V$ & $6.75 V$ & $18.14 mA$ & $30.65 mA$ & $47.7 mA$ \\
        \hline
    \end{tabular}
\end{center}
\vspace{0.2cm}
\textbf{Table 3:} Percent differences between the theoretical and measured values of the currents.
\begin{center}
    \begin{tabular}{|c|c|c|c|c|c|}
        \hline
        $I_1$ theoretical & $I_2$ theoretical & $I_3$ theoretical & $\%$ Diff $I_1$ & $\%$ Diff $I_2$ & $\%$ Diff $I_3$\\ 
        \hline
        $18.4 mA$ & $30.8mA$ & $49.3mA$ & $1.4\%$ & $3.3\%$ & $3.3\%$ \\
        \hline
    \end{tabular}
\end{center}
\section{Calculations}
Story time: I took a course in Linear Algebra several years back (or at least it feels that way). The theoretical part came to me extremely quickly, 
determinants and eigenvalues made perfect sense to me. However, on every test, I would always lose 5-10 points no matter what. The reason is, without fail, I would
always mess up my arithematic when row reducing. Almost certainly I would flip the signs or add two numbers incorrectly. Since then, I have developed
an avid hatred of row reduction by hand, and if there exists a way to avoid it,  I will. Luckily for us, Linear Algebra fills up our toolbox with plenty of tools, some of which include LU Decomposition, Cramer's Rule, and in the case
of a square matrix with a non-zero determinant (this is the only kind of matrix mathematicians are interested in anyway), matrix inversion. The last method is the one we choose for this matrix generated from the system of linear equations that follow
from Kirchhoff's Laws. Behold the beautiful mess that is inverting a matrix by hand— now you have the pleasure of experiencing
what Dr. Tran went through when she graded my tests!
\subsection*{Time for some Linear Algebra!}
Note: don't ask me why I choose to torture myself by doing this. I have plenty of free time (update: I spoke too soon) and got a good grade on the last lab report. It's the end of the year, I'm about to graduate from TAMS, and
society is on the cusp of complete collapse. 
I, quite literally, have nothing to lose. Without further ado, let us begin. 

We can arrange Equations \ref{eq:1}, \ref{eq:2}, and \ref{eq:3} in this fashion:
\begin{equation*}
    \begin{array}{lclclclcl}
        I_1 & + & I_2 & - & I_3 & = & 0\\
        R_1 I_1 & + & 0 & + &R_3 I_3& = & V_f\\
        0 & - & R_2 I_2 & - & R_3 I_3 & = & -V_0.\\
    \end{array}
\end{equation*}
From here it is clear that we can rewrite this system as the matrix equation $A\vec{x} = \vec{b}$, where 
\[
\begin{gathered}
A = \begin{bmatrix}
    1 & 1 & -1 \\
    R_1 & 0 & R_3\\
    0 & -R_2 & -R_3
\end{bmatrix}, \quad
\vec{x} = \begin{bmatrix}
    I_1\\
    I_2\\
    I_3
\end{bmatrix}, \quad
\vec{b} = \begin{bmatrix}
    0\\
    V_f\\
    -V_0
\end{bmatrix}
\end{gathered}
\]
Since we know the values of all the unknowns except for the currents, we rewrite the matrices and vectors with real number entries as such:
\[
\begin{gathered}
A = \begin{bmatrix}
    1 & 1 & -1 \\
    99.6 & 0 & 99.8\\
    0 & -326 & -99.8
\end{bmatrix}, \quad
\vec{x} = \begin{bmatrix}
    I_1\\
    I_2\\
    I_3
\end{bmatrix}, \quad
\vec{b} = \begin{bmatrix}
    0\\
    6.75\\
    -14.97
\end{bmatrix}
\end{gathered}
\]
A trivial calculation shows that $det(A)$ is non-zero and is equal to $74944.48$.
Therefore, we can solve for $\vec{x}$ using the equation
\begin{equation*}
    \vec{x} = A^{-1}\vec{b} \, ,
\end{equation*}
since it follows from the invertible matrix theorem that $A$ is invertible.
We want to find the exact value of $A^{-1}$. We have
\begin{equation*}
    A^{-1} = \frac{1}{det(A)}adj(A),
\end{equation*}
and we already know the value of $det(A)$. To find $adj(A)$, we first solve for the matrix of minors, utilize the "checkerboard" pattern (cofactor matrix), then take the resulting matrix's transpose to obtain the adjugate matrix.
Let us do this now. To find the matrix of minors, it's a fairly trivial computation that involves taking $9$ determinants. We omit the work for this step and present the matrix of minors here:
\begin{equation*}
\begin{pmatrix}
    32524.8 & -9940.08 & -32469.6 \\
    -425.8 & -99.8 & -326\\
    99.8 & 199.4 & -99.6
\end{pmatrix}
\end{equation*}
Now we "multiply" by the "checkerboard" matrix to obtain the matrix of cofactors. Clearly it is not actual matrix multiplication, hence the quotations around the word "multiply". We present the "checkerboard" matrix below:
\begin{equation*}
    \begin{pmatrix}
        + & - & + \\
        - & + & -\\
        + & - & +
    \end{pmatrix}
\end{equation*}
We multiply each respective entry $a_{ij}$ of $A$ by the respective $b_{ij}$ in the checkerboard matrix to get
\begin{equation*}
    \begin{pmatrix}
    32524.8 & 9940.08 & -32469.6 \\
    425.8 & -99.8 & 326\\
    99.8 & -199.4 & -99.6.
    \end{pmatrix}
\end{equation*}
Now we take the tranpose of this matrix to find the adjugate matrix, so we have $adj(A) =$
\[
\begin{pmatrix}
    32524.8 & 425.8 & 99.8 \\
    9940.08 & -99.8 & -199.4\\
    -32469.6 & 326 & -99.6
\end{pmatrix}
\]
Finally, keeping in mind that the determinant is a scalar, we divide the adjugate matrix by the determinant, so
\[
\begin{pmatrix}
    1 & 1 & -1 \\
    99.6 & 0 & 99.8\\
    0 & -326 & -99.8
\end{pmatrix}^{-1} \, \,  = \quad
\begin{pmatrix}
    0.434119 & 0.00568154 & 0.00133165 \\
    0.132633 & -0.00133165 & -0.00266064\\
    -0.433249 & 0.00434989 & -0.00132898.
\end{pmatrix}
\]
\newline
Now since
\begin{equation*}
    \vec{x} = A^{-1} \vec{b},
\end{equation*}
we have
\newline
\[
\vec{x} =
\begin{bmatrix}
    0.434119 & 0.00568154 & 0.00133165 \\
    0.132633 & -0.00133165 & -0.00266064\\
    -0.433249 & 0.00434989 & -0.00132898
\end{bmatrix}
\begin{bmatrix}
    0\\
    6.75\\
    -14.97
\end{bmatrix} = 
\begin{bmatrix}
    0.0184156\\
    0.0308411\\
    0.0492566
\end{bmatrix}
\]
Taking note that the values obtained are in Amps, we conclude that
\[
\vec{x} =
\begin{bmatrix}
    I_1\\
    I_2\\
    I_3
\end{bmatrix}
=
\begin{bmatrix}
    18.4 mA\\
    30.8 mA\\
    49.3 mA
\end{bmatrix},
\]
so $I_1 = 18.4 mA$, $I_2 = 30.8 mA$, and $I_3 = 49.3 mA$.

You thought the calculations section was over, but it was me DIO! In this case, DIO represents the percent differences. Some quick applications of Equation 
\ref{eq:6} yields
\begin{equation*}
    \text{Percent Difference} = \frac{\mid \! 18.4-18.14  \mid}{\frac{\mid 18.4+18.14 \mid}{2}} \cdot 100\% = 1.42\%
\end{equation*}
for $I_1$,
\begin{equation*}
    \text{Percent Difference} = \frac{\mid \! 30.8-30.65  \mid}{\frac{\mid 30.8+30.65 \mid}{2}} \cdot 100\% = 0.49\%
\end{equation*}
for $I_2$, and 
\begin{equation*}
    \text{Percent Difference} = \frac{\mid \! 49.3-47.7  \mid}{\frac{\mid 49.3+47.7 \mid}{2}} \cdot 100\% = 3.30\%
\end{equation*}
for $I_3$. Our fruitless toil finally comes to an end.
\section{Discussion of Results and Error Analysis} 
There are several possible sources of error for this experiment— we begin by discussing the resistors. 
The theoretical values were calculated with the measured values of the resistors which reduced the error, however
they will always fall somewhere below a $\pm 5\%$ error due to the nature of resistors. The gold band on the resistor indicated the error bound of 
$\pm 5\%$. Another source of error includes the equipment used, unfortunately we don't have access to professional tier research equipment. 
As a result, there will be many small sources of error accumulating throughout the experiment, such as the DMM measuring values slightly off, the quality of the circuit board affecting the voltages, 
the quality of the resistors, and so on.

However, even with all these sources of error, our results were quite splendid. Truly, we live in the best of all possible worlds. 
The average percent error of $1.74\%$ indicates that our results stayed quite true to the theory, a concept which I will re-iterate in the following section.
\section{Conclusion}
Finally this lab report comes to an end. We have done many things, like proving some interesting theorems, doing some neat calculations, and looking at some cool circuits. 
We used Kirchhoff's Laws to analyze such aforementioned cool circuit(s), and measured the voltages and calculated the current values by such measured voltages.
Our results seem to support our claim. The average percent error of $1.74\%$ indicates that our results stayed quite true to the theory. Thus, we have supported the amazing claim that
Kirchhoff's Laws seem to apply in the case of this particular circuit.\footnote{Note how I didn't say anything about how this result supports the validity of Kirchhoff's Laws. Indeed, we could verify it manually for 999,999 circuits and if one failed to hold, the entire claim was false from the premise. Truly, if we were to show the validity of Kirchhoff's Laws, we would have to provide some form of formal proof that works for an arbitrary circuit. Wherever could that be...}
\footnote{Please excuse this mathematician's dry sense of humor, my sincerest apologies for making you read this (this statement applies to the entire lab report). Thanks in advance.} 
\end{document}
